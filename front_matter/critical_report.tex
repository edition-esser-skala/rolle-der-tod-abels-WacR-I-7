\documentclass[tocstyle=ref-genre]{ees}

\firstname{\makebox[9cm][l]{Johann Heinrich\hfill Johann Michael}}
\lastname{\makebox[9cm][l]{Rolle\hfill \&\hfill Haydn}}
\shortname{Rolle \& Haydn}

\begin{document}

\eesTitlePage

\eesCriticalReport{
  –    & –   & org         & Bass figures appear in the following movements (bars in parentheses): 1.2, 1.4, 1.6 (20), 1.8, 1.10, 1.12, 1.14, 1.16, 1.18, 2.2, 2.4, 2.6 (23–49, 61 and 73–75), 2.8, 2.10 The remaining bass figures were added by the editor. \\
  1.1  & 6   & ob, fag, vl 2 & grace note missing in \B1 \\
       & 7   & ob, fag, vl & grace note missing in \B1 \\
       & 10  & ob 1        & 1st \halfNote\ in \B1: \flat a′2 \\
       & 16  & ob 2, fag 1, vl 2 & grace note missing in \B1 \\
       & 17  & A           & grace note missing in \B1 \\
       & 33  & ob 2, vl 2  & grace note missing in \B1 \\
       & 34  & A           & grace note missing in \B1 \\
       & 50  & fag         & grace note missing in \B1 \\
       & 51  & A           & grace note missing in \B1 \\
       & 52  & A           & 1st \halfNote\ in \B1: g′2 \\
  1.4  & 51  & vla         & 1st \halfNote\ in \B1: \flat e′2 \\
  1.7  & 22  & fl, fag     & rhythm of 5th \eighthNote\ in \B1: \sixteenthNote–\sixteenthNote \\
       & 29  & fag 2, vl, Kain & grace note missing in \B1 \\
       & 31  & fl 2        & grace note missing in \B1 \\
       & 34  & fl 1        & 1st \quarterNote\ in \B1: g″16–\flat  e′′′16 \\
       & 41  & fl 2        & grace note missing in \B1 \\
  1.9  & 16  & cor 2       & grace note missing in \B1 \\
       & 20  & cor         & grace note missing in \B1 \\
       & 64  & vl 2        & grace note missing in \B1 \\
       & 116 & vla         & bar in \B1: g′4–g′8 \\
  1.10 & –   & org         & upper voice after \B1, lower voice after \C1 \\
  1.11 & 8   & vl 2        & bar in \B1: e″4.–d″8 \\
       & 17  & vl 2        & bar in \B1: e′4.–d′8 \\
  1.13 & 82  & vla         & grace note missing in \B1 \\
       & 86  & vla         & grace note missing in \B1 \\
       & 163 & vl 2        & grace note missing in \B1 \\
       & 165 & vl 2        & grace note missing in \B1 \\
       & 215 & vl 2        & 1st \eighthNote\ in \B1: \sharp f″8 \\
       & 229 & vl 1        & grace note missing in \B1 \\
       & 233 & vl          & grace note missing in \B1 \\
  1.14 & 63  & vl 1        & grace note missing in \B1 \\
  \midrule
  2.1 & 9    & vla         & 1st \eighthNote\ in \B1: d′8 \\
  2.3 & 50   & vla         & grace note missing in \B1 \\
  2.7 & 21   & vl 2        & grace note missing in \B1 \\
      & 42   & vla         & grace note missing in \B1 \\
  2.8 & –    & timp        & voice only appears in \B2 \\
  2.9 & –    & timp        & voice only appears in \B2 \\
  2.10a–e & – & –          & These movements (“letzter Aufzug”) have been written by Johann Michael Haydn (\A1) to extend the story line of Thirza. Instrument names are missing throughout the source. \\
}

\eesToc{
% Thirza (S), Hamiel (S), Sunam(S), Eva (A), Mehala (A), Abel (T), Adam (B), Kain (B)

\part{erstertheil}

\begin{movement}{lobtden}
  \voice[Chor]
  Lobt den Herrn! Lobt den Herrn!\\
  Die Morgenſonne weckt die Flur aus ihrer Ruh;\\
  und der ganzen Schöpfung Wonne\\
  ſtrömt verjüngt uns wieder zu.\\

  Lobt den Herrn! Lobt den Herrn!\\
  In frühen Düften lobet ihn der Blumen Flor;\\
  auf den Wipfeln, in den Lüften\\
  ſinget ihn der Vögel Chor.\\

  Lobt den Herrn! Lobt den Herrn!\\
  Aus ſeiner Höhle brüllt das Wild ihm ſeinen Dank.\\
  O! vor allen, meine Seele,\\
  tön ihm früh dein Lobgeſang!
\end{movement}

\begin{movement}{siesingen}
  \voice[Kain]
  Sie ſingen!\\
  Ha, gewiß ein neues Lied des Lieblings,\\
  dem allein die heitre Freude blüht,\\
  der müßig bey der Heerde ſitzet,\\
  wenn dieſe Stirn von ſchwerer Arbeit ſchwitzet.\\
  Seh ich nicht in der Laub auch Adam?\\
  Wie entzückt umarmt er ihn!\\
  Mit welcher Liebe blickt ihn Eva an!\\
  Wie ihre Herzen ſich ergießen!\\
  Wie ihre Freudenthränen fließen!\\
  Ach, immer heller wird mein dunkles Traumgeſicht.\\
  Ich ſeh, ich ſeh, es täuſcht mich nicht:\\
  Mein iſt der Fluch und ſein der Seegen.\\
  Sah ich im Traum auf Blumenwegen\\
  nicht alle Kinder Abels gehn?\\
  In ihrem Thal war die Natur nur ſchön.\\
  In dunklen Schatten grüner Lauben,\\
  in Feigenhaynen, unter Trauben,\\
  umkränzt mit Roſen ſaßen ſie,\\
  und goldne Frucht fiel ohne Müh in ihren Schoos,\\
  nur ſanfte Lieder ſangen ſie.\\
  Ein dornicht Feld, ein Thal, wo Armuth wohnet,\\
  und Seegen nie den Fleiß belohnet,\\
  war meiner Kinder Theil.\\
  Und ach! mit welcher Quaal,\\
  erinnert ſichs mein Geiſt!\\
  in dieſes Dornenthal fiel in der Nacht\\
  die Schaar von Abels Söhnen.\\
  Noch ſeh ich meine Kinder höhnen,\\
  noch ſchallet ihr Geſchrey\\
  mir ſchrecklich in mein Ohr,\\
  noch ſteigt vor mir die Flamme hoch empor\\
  von Hütten, die der Feind entzündet,\\
  noch ſeh ich, wie man ſie als Sclaven bindet,\\
  hört nicht ihr Flehen, nicht ihr Schreyn,\\
  und führt ſie weg, dienſtbar zu ſeyn.
\end{movement}

\begin{movement}{ichelend}
  \voice[Kain]
  Ich elend! elend meine Kinder!\\
  Gewitter Gottes, trefft mich Sünder!\\
  Abgrund, eröffne dich! verſchlinge mich!\\
  Umſonſt iſt dieſes Flehen,\\
  der Richter läßt es nicht geſchehen.\\
  Elender, du mußt elend ſeyn!\\
  ſo waffne dich, und ſey es nicht allein!
\end{movement}

\begin{movement}{seymir}
  \voice[Adam]
  Sey mir gegrüßt, mein erſtgebohrner Sohn!\\
  ach, daß ich dieſen Trauerton\\
  von deinen Lippen nicht gehöret,\\
  der Pein in meinem Buſen nähret!\\
  Die führt mich zu dir, dieſe Pein.

  \voice[Kain]
  Nicht Liebe, die gehört dem Abel nur allein.

  \voice[Adam]
  Ja Kain, zärtliche beſorgte Liebe.\\
  Was nähreſt du für finſtre Triebe,\\
  für ſchwarzen Groll in deiner Bruſt?\\
  Groll gegen den, der unſre Luſt\\
  durch ſeinen heitern Reiz der Jugend,\\
  durch ſeine Andacht, ſeine Tugend\\
  und ſeine ſanften Lieder iſt!\\
  O du! der du mein Erſtgebohrner biſt!\\
  Mein Sohn, mein Kain!\\
  quäle mit dieſem Ungeſtüm nicht deine Seele.\\
  Lieb ihn, wie er dich liebt,\\
  erheitre dein Geſicht.

  \voice[Kain]
  Wie Abel lächeln kann ich nicht,\\
  gebieten kann ich nicht dem Ernſt,\\
  der euch verdrießet,\\
  daß er in Thränen ſanft zerfließet.

  \voice[Adam]
  Das iſt kein männlicher, kein ernſter Sinn,\\
  nein, Unzufriedenheit und Gram reißt dich dahin,\\
  dein freudenloſes finſteres Betragen zeigt es,\\
  und die itzt ausgeſtoßnen Klagen.

  \voice[Kain]
  Vielleicht ſind auch mehr Freuden noch für dich.\\
  Die größte Laſt des Fluches fiel auf mich,\\
  den Erſtgebohrnen;\\
  nur ich bin verlohren,\\
  nur ich bin vom Weibe zum Elend gebohren!

  \voice[Adam]
  Ach Sohn! was hört mein Ohr!\\
  Du läſterſt! Kein Geſchöpf ruft Gott zum Elend hervor.\\
  Laß dich Vernunft und Tugend rühren,\\
  ſie werden Freuden dir in deine Seele führen.\\
  Mein Sohn! ach höre mich!\\
  bey dieſen Thränen bitt ich dich,\\
  lieb Abeln! du wirſt uns entzücken,\\
  und wie wird er an ſeine Bruſt dich drücken!

  \voice[Kain]
  Ich haß ihn nicht. Doch, was euch alle rührt,\\
  der Weichlichkeit, die mir dein Herz entführt,\\
  die Thränen euch entlockt,\\
  der hab ich Haß geſchworen,\\
  dadurch haſt du das Paradies verlohren.
\end{movement}

\begin{movement}{owort}
  \voice[Adam]
  O Wort, dafür mein Geiſt erzittert,\\
  das wie ein Donner mich erſchüttert.\\
  O Vorwurf voller Todespein,\\
  mit tauſend Stacheln gräbt er\\
  in mein Herz ſich ein!\\
  Welche Ahndung faſſet mich!\\
  Also werden Enkel dich\\
  in der Erde fernſten Tagen,\\
  Adam, Adam, laut verklagen,\\
  Rettung aus dem Elend ſuchen,\\
  aber keine Rettung ſehn!\\
  über deinem Haupte ſtehn\\
  und dir erſten Sünder fluchen!
\end{movement}

\begin{movement}{wieseufzet}
  \voice[Kain]
  Wie ſeufzet er!\\
  wie kläglich ringt er\\
  über ſeinem Haupt die Hände!\\
  Mein nagender qualvoller Vorwurf\\
  dringt ihm tief ins Herz.\\
  Ach, Kain! ach, Kain!\\
  wende von diesem Ungeſtüm\\
  dich zur Vernunft zurück.\\
  Kannſt du des frommen Vaters Blick\\
  voll ſolcher Seelenangſt ertragen,\\
  ſein Seufzen hören, und ſein Klagen!\\
  Noch biſt du nicht ganz deines Elends Raub,\\
  eil, und wirf dich vor ihm hin in den Staub!
\end{movement}

\begin{movement}{meinvater}
  \voice[Kain]
  Mein Vater, ach! verzeihe!\\
  Sieh dieſe Thränen meiner Reue,\\
  ich ſehe das, was ich gethan,\\
  mit Schauer und Entſetzen an.\\
  Ach Vater, fluche nicht dem Sohne,\\
  von dir zu des Allmächtgen Throne,\\
  zu meinem Bruder will ich gehn,\\
  Vergebung zu erflehn.
\end{movement}

\begin{movement}{ogott}
  \voice[Adam]
  O Gott! mein Blick ſchaut dankbar zu dir auf!\\
  Heut erhöreſt du mein Sehnen.\\
  Mein Kain! o, mein Sohn, ſteh auf!\\
  Mit Wohlgefallen ſieht Gott dieſe Thränen.\\
  Seyd mir geſegnet, frohe Stunden!\\
  Ich habe meinen Erſtgebohrnen wieder funden.\\
  Er bringt mit dieſem thränenvollen Blick\\
  uns Freuden, Fried und Ruh zurück.\\
  O Sohn! wie wird Gott deiner ſich erbarmen,\\
  dich zögre nicht, komm, laß den Bruder dich umarmen!

  \voice[Abel]
  Du liebeſt mich, mein Bruder!\\
  du liebſt mich! ach! ſag es mir,\\
  daß ich von deinen Lippen es vernehme!

  \voice[Kain]
  Ich liebe dich!\\
  ja, ich Elender ſchäme des Unrechts mich,\\
  daß ich dir dieſes Herz verſagt,\\
  die Ruh ſo lang von euch verjagt,\\
  mit Unmuth eure Tag erfüllet,\\
  und mich in Trübſinn eingehüllet.\\
  Schnell hebt ſich meine Seel empor,\\
  und geht aus ihrer Nacht hervor.\\
  Mein Bruder! du kannſt mir vergeben,\\
  und ſieheſt nicht zurück in das vergangne Leben.
\end{movement}

\begin{movement}{wennderjunge}
  \voice[Abel]
  Wenn der junge Tag erwacht,\\
  ſo verſchwinden alle Sorgen,\\
  die ein leichter Traum am Morgen\\
  auf dem Lager uns gemacht.\\
  O Kain, Kain! mein Entzücken\\
  vermag ich dir nicht auszudrücken,\\
  der Ton erſtirbt für ſüße Luſt;\\
  nur drücken kann ich dich an dieſe Bruſt.
\end{movement}

\begin{movement}{okinder}
  \voice[Eva]
  O Kinder! ſeit das Paradies verſchwunden,\\
  hab ich nicht ſolche Freud empfunden,\\
  als dieſer Anblick mir gewährt,\\
  da Fried und Eintracht wiederkehrt.\\
  Ach, Adam! die, die wir erzeugten,\\
  die lieben ſich!\\
  Nichts iſt meiner Wonne gleich.\\
  Umarmt, geliebten Kinder, euch.\\
  Die Thränen, die aus euren Augen fließen,\\
  will ich von euren Wangen küßen.
\end{movement}

\begin{movement}{achschwester}
  \voice[Mehala]
  Ach Schweſter! ſing in meine Lieder!\\
  heut kommt die ſanfte Ruh mir wieder.

  \voice[Thirza]
  Mehala! wie der Lenz erquicket,\\
  hat dieſer Anblick mich entzücket.

  \voice[Mehala]
  Du biſt mir ſchöner nun, Natur!

  \voice[Thirza]
  Du blühſt mir lieblicher, o Flur!

  \voice[Mehala]
  Dein Licht iſt heller, ſtiller Mond!

  \voice[both]
  Nun Ruh in unſern Hütten wohnt.

  \voice[Thirza]
  O paradieſisch große Freude,\\
  die beſten Blumen laßt uns beyde\\
  in unſre Laube ſtreun.

  \voice[Mehala]
  O paradieſisch ſchönes Leben!\\
  die beſte Frucht vom Baum und Reben\\
  ſoll unſer Herz erfreun.

  \voice[both]
  Und dieſer Tag ein Feſt uns ſeyn.
\end{movement}

\begin{movement}{meinbruder}
  \voice[Abel]
  Mein Bruder! unſerm Gotte, der uns liebt,\\
  der meinen Bruder heut mir wiedergiebt,\\
  will ich an meinem Altar danken.\\
  Haſt du nicht auch, Geliebter, den Gedanken?\\
  Willſt du nicht auch zu deinem Altar gehn?\\
  Das ſchönſte Lamm hab ich erſehn,\\
  dem Herrn zum Opfer anzuzünden.\\
  Auch du, mein Bruder, wirſt ein Opfer finden,\\
  dem Herren angenehm.\\
  Verſiegle beym Altar den Bund,\\
  der ſtets der Wunſch von meinem Herzen war.

  \voice[Kain]
  Ich will es thun,\\
  und auch dem Herrn ein Opfer bringen.\\
  Zwar wenig kann ich nur erzwingen,\\
  das, was des Feldes Armuth giebt.

  \voice[Abel]
  Du weißt, Geliebter, daß Gott nicht das Opfer liebt,\\
  den Opfernden liebt er;\\
  er achtet nicht auf das Lamm, das man ihm ſchlachtet,\\
  nicht auf die Frucht, die eine Flamm verzehrt,\\
  wenn Lieb und Andacht nur im Herzen ihn verehrt.

  \voice[Mehala]
  Geliebter, zürne nicht! Ich ſeh die Zeichen\\
  des finſtern Grams zurück auf deine Stirne ſchleichen,\\
  ach, kämpfe, daß er nicht dein Herz\\
  erfülle, uns aufs neu zum Schmerz!

  \voice[Kain]
  Sey ruhig; er ſoll nicht mein Herz erfüllen,\\
  mit Dunkel eure Tage nicht umhüllen.\\
  Zwar ihr beleidigt mich ins Angeſicht\\
  mit eurem Uebermas von Freuden:\\
  So laſterhaft war Kain nicht,\\
  als ihr dadurch mich macht.\\
  Doch, ſchon gewohnt zu leiden,\\
  erduld ich es.\\
  Wer mit des Feldes Arbeit ringt,\\
  bezwingt auch leichten Gram,\\
  der zu dem Herzen dringt.\\
  Mein Opfer ſoll auf zu dem Herren wallen,\\
  er thu mit mir nach ſeinem Wohlgefallen.

  \voice[Adam]
  Mein Sohn! laß es ein frohes Opfer ſeyn.\\
  Schau, alles ruft dich auf, dich mit uns zu erfreun.\\
  Mit Unmuth iſt vor Gott von uns niemand erſchienen;\\
  wir ſollen ihm mit Freuden dienen.
\end{movement}

\begin{movement}{frohgeht}
  \voice[Mehala]
  Froh geht dir die Sonne auf,\\
  froh vollendet ſie den Lauf!\\
  Feld und Flur im Blumenkleide\\
  und ſelbſt Arbeit giebt uns Freude.\\
  Opfert mit dem frohſten Triebe,\\
  und verſiegelt vor dem Herrn\\
  euren neuen Bund der Liebe,\\
  frohe Herzen hört er gern.
\end{movement}

\begin{movement}{sokomm}
  \voice[Abel]
  So komm und reiche mir die Hand!\\
  dieß Opfer knüpfe unſer Band\\
  feſt vor dem Herrn.\\
  Ich ſchwöre beym Blut des Opferlamms,\\
  bey dieſer treuen Zähre,\\
  dir ewig meine Zärtlichkeit,\\
  die nichts als deine Lieb erfreut.
\end{movement}

\begin{movement}{achliebe}
  \voice[Abel]
  Ach, liebe mich ſo wie ich dich!\\
  Laß mit den reinſten Trieben\\
  uns ſo wie Engel lieben!\\
  ſo werden Engel mit uns gehn,\\
  mit uns bey dem Altare ſtehn:\\
  ſo wird Gott ſelbſt uns da begegnen,\\
  und uns allda ſein Antlitz ſegnen.
\end{movement}

\begin{movement}{siegehn}
  \voice[Adam]
  Sie gehn – doch Kain nicht erfreut.\\
  Bang, bang iſt mir!\\
  ach! fleht, daß Gott den Gram zerſtreut,\\
  den wir in ſeinem Herzen merken!\\
  vielleicht wird ihn ſein Opfer ſtärken.
\end{movement}

\begin{movement}{weltrichter}
  \voice[Chor]
  Weltrichter! der du uns gerichtet,\\
  doch nicht den Sünder ganz vernichtet,\\
  erbarme dich!\\
  Groß iſt die Verheißung, die du uns gethan.\\
  Nimm darum, Erbarmer, die Opfernden an!\\
\end{movement}

\begin{movement}{mehala}
  \voice[Eva]
  Mehala! Thirza! alle meine Kinder!\\
  gerecht iſt Gott, doch gnädig auch dem Sünder;\\
  er wirds auch Kain, meinem Erſtgebohrnen, ſeyn.\\
  Das Opfer ſelbſt wird ſeinen Gram zerſtreun.\\
  Des Herren Gnade wird ſein enges Herz erweitern,\\
  und ſeine Seel erheitern.\\
  Auch Adam hoffet noch.\\
  Erheitert das Gemüth, ihr Töchter,\\
  und ſinget mir indeſſen euer Lied,\\
  das Abels Lob erzählt, das Lob des Frommen,\\
  bis beyde Hand in Hand zurücke kommen.
\end{movement}

\begin{movement}{frommist}
  \voice[Mehala]
  Fromm iſt Abel, der Hirt,\\
  führt er im Thale die Heerden,\\
  ſo iſt Gott ſein Gedank\\
  und der Schöpfer ſein Lied.

  \voice[Thirza]
  Weiſ’ iſt Abel, der Hirt,\\
  ins ſanfte Lächeln der Augen\\
  miſcht ſich denkender Ernſt,\\
  Seele redet im Blick.

  \voice[Mehala]
  Schön iſt Abel, der Hirt,\\
  ſieh, braune ſchattigte Locken\\
  kräuſeln ſich um die Stirn,\\
  fließen die Schultern herab.

  \voice[Thirza]
  Fromm iſt Abel, der Hirt,\\
  wenn er vom Ewigen ſinget;\\
  o, dann wallet mein Herz\\
  von Empfindungen voll.

  \voice[Mehala]
  Weiſ’ iſt Abel, der Hirt,\\
  ſind Gottes Wege mir dunkel,\\
  wie enthüllet er ſie,\\
  wie zerſtreut er die Nacht!

  \voice[Thirza]
  Schön iſt Abel, der Hirt,\\
  lang iſt er und reizend gebildet.\\
  Aus der ſchlanken Geſtalt\\
  ſchimmert der Engel hervor!
\end{movement}

\part{zweytertheil}

\begin{movement}{sehtdort}
  \voice[Chor]
  Seht! dort ſteigt der Opferrauch herauf!\\
  von Abels Altar ſteigt er auf!\\
  Still feyert die Natur,\\
  als wäre Gott zugegen.\\
  Die Winde ruhn,\\
  es regen ſich die Gebüſche nicht!\\
  Ein angenehmer Duft\\
  ſtrömt von dem Altar her,\\
  und füllt die ganze Luft.\\
  Der Richter hat das Opfer von dem Frommen,\\
  er hat es gnädig, gnädig aufgenommen.

  Ach weh! in welche Nacht verſinkt die Flur,\\
  wo Kains Altar ſteht!\\
  Ein ängſtlich Rauſchen tönt durch die Natur!\\
  Ein Sturmwind heult und weht das Opfer weg,\\
  umhüllt den Opfernden mit Rauch,\\
  und füllt die Flur umher mit ſchwarzem Dampf.\\
  Ach weh! verworfen iſt der Haſſende.
\end{movement}

\begin{movement}{achmeine}
  \voice[Adam]
  Ach, meine Eva! welch ein Schmerz,\\
  ſchreckliche Beſorgnis füllt mein Herz.\\
  Er, den ſein Zuſtand ſtets betrübet,\\
  der ſtets geglaubt, nur Abel ſey geliebet,\\
  ihm lächle der Allmächtge nur,\\
  ihm blüh allein die Flur,\\
  der nicht vermocht, dem Argwohn zu gebieten:\\
  was wird er thun? wie wird er wüthen?\\
  O Sünde! welche Nacht\\
  von Elend haſt du über uns gebracht.\\
  Ach, ich muß gehen,\\
  ich muß ſelbſt den Verworfnen ſehen.

  \voice[Eva]
  Ich folge dir; voll Ahndung iſt dies Herz,\\
  voll banger Ahndung und voll Schmerz!\\
  Mehala! Thirza! bleibet, meine Kinder,\\
  und betet für den Sünder!\\
  Ich will ihn ſuchen!\\
  Ach, er iſt vielleicht entflohn,\\
  fern von uns weg, ach Kain! ach mein Sohn!

  \voice[Thirza]
  Mehala! Angſt ſitzt tief in unſern Herzen.\\
  Ich leſ’ in deinem Auge Schmerzen;\\
  und ich, nie fühlt ich ſolche Pein.\\
  Ihr Freuden, ſolltet ihr ſo kurz nur ſeyn!\\
  Ach hätte Kain uns betrogen,\\
  und ſein Geſicht die Liebe nur gelogen,\\
  die er verſprach?

  \voice[Mehala]
  Ich fürcht es, Schweſter! ach, ich fürcht es nun!\\
  mein Herz ſang der Verſöhnung Lieder.\\
  Nun kehrt mein langer Kummer wieder.\\
  Der ſchwarze Unmuth ſitzt zu tief in ſeiner Bruſt.\\
  Nichts war in der Natur für ihn ein Quell der Luſt.\\
  Doch hofft ich noch, er würde wiederkehren,\\
  und Gott würd unſer Flehn erhören.\\
  Da er vor Adam thränend lag,\\
  dach ich: Heut iſt der Tag,\\
  der meinen Wunſch erfüllt!\\
  Doch bald ſah ich die Zeichen des finſtern Grams\\
  zurück auf ſeine Stirne ſchleichen.\\
  Nun ihm der Herr ganz ſeine Gnad entzieht –\\
  gerecht iſt Gott – wer weiß, wohin ſein Fuß entflieht!
\end{movement}

\begin{movement}{wieeine}
  \voice[Mehala]
  Wie eine Blume ſinket,\\
  die keinen Thau mehr trinket,\\
  kein Sonnenſtrahl erquickt;\\
  ſo ſink ich unter Schmerzen,\\
  die dem beklemmten Herzen\\
  der Hoffnung Troſt entrückt.
\end{movement}

\begin{movement}{achgott}
  \voice[Adam]
  Ach Gott! ach Abel!

  \voice[Eva]
  Adam, wo bin ich?\\
  Eiskalter Schauer faſſet mich!\\
  Da liegt er!\\
  Blut fließt von der Stirne nieder!\\
  Ach Abel! beſter Sohn! erwache wieder!\\
  Adam! liegt er nicht hier?\\
  trugſt du ihn her?\\
  ſprich, liegt er nicht vor mir?

  \voice[Adam]
  Ach! ihre Sinnen ſind zerrüttet!\\
  mit welchem Jammer, Gott,\\
  haſt du mich überſchüttet!\\
  nicht du – nein ich;\\
  ach das iſt meiner Sünde Lohn!

  \voice[Thirza]
  Mein Vater!

  \voice[Mehala]
  Eva!

  \voice[Eva]
  Abel! ach mein Sohn!

  \voice[Thirza]
  O Schweſter, welch ein Klageton!\\
  ſie ſehn uns nicht. Ach, ich muß gehen\\
  und ſelbſt die ſchreckliche Geſchichte ſehen.

  \voice[Adam]
  Geliebter Abel! du, du todt!\\
  mein andrer Sohn! o Gott!\\
  ein Abſcheu der Natur!\\
  Ach, meine Glieder beben!\\
  Allmächtiger, du wirſt die Klagen uns vergeben.

  \voice[Eva]
  Wie liegt die Hülle da im Gras!\\
  mit Blut befleckt, die Wange blaß!\\
  dieß ſtarke Aug weint nicht mehr Freudenzähren!\\
  Der Mund wird uns kein Lied mehr lehren.\\
  Todt iſt er! ach, iſt das der Tod?\\
  Wie ſchrecklich iſt er, Gott!\\
  ach Abel! todt biſt du, erſchlagen!

  \voice[Kain]
  Ja, ich hab ihn erſchlagen,\\
  vor dieſem Donner bebt,\\
  Weib! ich hab ihn erſchlagen!
\end{movement}

\begin{movement}{welchwinseln}
  \voice[Kain]
  Welch Winſeln ſchlägt mein Ohr?\\
  Welch Seufzen ſteigt aus dem Gebüsch hervor?\\
  Es rieſelt hinter mir als wie ein Bach!\\
  das iſt ſein Blut!\\
  Es fließt mir nach, das iſt ſein Blut!\\
  Er iſt es – röcheln hör ich ihn!\\
  Wohin flieh ich? wohin?
\end{movement}

\begin{movement}{entsetzen}
  \voice[Mehala]
  Entſetzen – Kain – mein Mann – erſchlug ihn!\\
  hat den Brudermord gethan!\\
  Entſetzen – welch Verbrechen!\\
  Ach, – meine Mutter!\\
  wer vermag es auszuſprechen,\\
  was deine Bruſt zerreißt?\\
  Doch, fluch ihm nicht, ach, Adam!\\
  Sieh, wie die Hölle ſchon in ſeinen Buſen dringet!\\
  und er mit der Verzweiflung ringet!\\
  wie jagt ſie ihn!\\
  Ach, Kain! einſam wilſt du in die Welt entfliehn!\\
  einſam und hülflos und verlaſſen!\\
  ich folge dir, ich kann, ich darf nicht haſſen.

  \voice[Adam]
  Ach, Eva – ſtarr ſieht ſie,\\
  wen ſucht dein Auge?

  \voice[Eva]
  Iſt er fort, der uns geflucht?\\
  wo iſt er hin? – ich muß ihn ſuchen,\\
  und ſagen, Adam nicht zu fluchen!\\
  ich ſündigte zuerst, mich treffe Fluch und Wuth!\\
  mich klag es an, dieß Blut.

  \voice[Adam]
  O, welche Quaal machſt du dem Herzen!\\
  Ach, ich beſchwöre dich bey unſern Schmerzen,\\
  von dieſem Vorwurf gegen dich laß ab!\\
  wir beyde ſündigten; Gott ſieht auf uns herab,\\
  gedenke ſeines Worts voll Segen,\\
  die Tugend führt der Tod\\
  dem ewgen Lohn entgegen.

  \voice[Thirza]
  Ach, Elend! – er erwachet nicht, mein Abel –\\
  Er, mein Glück, mein Leben,\\
  und ich ſeh das verhaßte Licht,\\
  und nichts kann mir ihn wieder geben!\\
  Ach, Thränen, fließt in meinen Schmerz,\\
  erleichtert das beklemmte Herz!\\
  Er todt! – mein Abel todt!\\
  auf ewig mir entriſſen! –\\
  warum mußt ich nicht einmal noch ihn küſſen. –\\
  Wie liebreich hätt’ er mich im Sterben angeblickt!\\
  Dann hätte meinen Geiſt ſein Hauch in ſich geſogen!\\
  Dann hätt’ er mich mit ſich der Erd’ entrückt,\\
  mit ihm wär ich zum Himmel aufgeflogen.\\
  Ach, ſüßer Gedanke, heißes Sehnen!\\
  erweich dieß Herz zu Thränen.
\end{movement}

\begin{movement}{fliesst}
  \voice[Thirza]
  Fließt unaufhaltſam hin, ihr Zähren!\\
  fließt hin in meinen Jammerton.\\
  Nicht ſeinen Abſchied ſollt ich hören,\\
  ſo ſchnell iſt mir ſein Geiſt entflohn!\\
  Ihr grünen Lauben werdet fragen:\\
  wo iſt dein treuer Jüngling hin?\\
  Ihr Quellen werdet um ihn klagen,\\
  und ſeufzen, ach, wo iſt er hin?\\
  Er wird nicht mehr bey euch erſcheinen,\\
  und ich nur einſam bey euch weinen!
\end{movement}

\begin{movement}{achtochter}
  \voice[Eva]
  Ach Tochter, du zerreißeſt dieſes Herz!\\
  wie martert mich dein Schmerz!\\
  ich fühl, ich fühle deine Klagen.\\
  Vorwürfe ſind es die mich nagen,\\
  mich, die ich dieſe Trauernacht,\\
  und Fluch und Tod auf euch gebracht.

  \voice[Adam]
  Hörſt du den Donner?\\
  Gott wird kommen zu fodern\\
  das vergoßne Blut des Frommen.\\
  Ach, Eva! laß uns zu dem Richter flehn,\\
  vielleicht erbarmt er ſich des Fliehenden.
\end{movement}

\begin{movement}{herrwende}
  \voice[Adam]
  Herr, wende nicht dein Angeſicht\\
  von dem Verbrecher, tödt ihn nicht!

  \voice[Eva]
  Herr, du vergabſt den erſten Sündern,\\
  vergieb, vergieb auch ihren Kindern.

  \voice[Adam]
  Laß ihn, wenn er wird reuend flehen,\\
  vor deinem Zorne nicht vergehen.

  \voice[Eva]
  Laß ihm, wird er um Gnade weinen,\\
  in ſeinem Jammer Troſt erſcheinen.
\end{movement}

\begin{movement}{Ertoedtet}
  \voice[Eva]
  Er tödtet, ach! er tödtet ihn.\\
  Auch Kain iſt nicht mehr!\\
  Hörſt du die Schläg?\\
  O, ſchrecklich iſt der Herr,\\
  wenn er ſich zum Gericht erhebet.\\
  Ach, Adam, mein Gebein erbebet,\\
  durch meine Glieder fährt ein Todesſchauer hin.\\
  Ich bin beyder Kinder Mörderinn.

  \voice[Adam]
  Nein, Eva, nein, der Herr erhört gewiß uns Armen,\\
  der Gnadenvolle wird ſich ſein erbarmen!

  \voice[Mehala]
  Ja, Gnaden voll iſt Er!\\
  ich eile zu euch her,\\
  euch mit der Nachricht Troſt zu geben,\\
  er lebet, Kain, und ſoll leben.\\
  Ich eilt ihm nach, doch ich erreicht ihn nicht,\\
  in dem Gebüſch verlohr ich ihn aus dem Geſicht.\\
  Ich rief ihn, irrte hin und wieder,\\
  da ſenkte ſchnell ſich eine ſchwarze Wolke nieder,\\
  es donnerte, und Feuer brach hervor.\\
  Ich ſank zur Erd’ und hob mein betend Aug’ empor,\\
  und aus der Wolken hört ich eine Stimme,\\
  die ſchien, als wie im Grimme,\\
  dem Flüchtigen das Fliehen zu verbieten:\\
  Wo iſt dein Bruder? donnert ſie.\\
  Ich weiß es nicht, ſoll ich ihn hüten?\\
  antwortet er verwirrt. –\\
  So ſchauerte ich nie. –\\
  Steh, rief die Stimm, und hör dein Urtheil an:\\
  Was, Kain! was haſt du gethan?\\
  Sieh, deines Bruders Blut\\
  ſchreyt zu mir von der Erde,\\
  daß es gerächet werde.\\
  Es ſey gerächt! dich martre Höllenpein;\\
  unſtät und flüchtig ſollſt du ſeyn.\\
  Da jammert er und rief mit Beben:\\
  ſo bin ich denn verflucht,\\
  ſo raubt mein Leben,\\
  wenn nun mein Fuß unſtät und flüchtig irrt,\\
  der erſte der mich finden wird!\\
  Nein, rief der Richtende:\\
  Gewiſſensangſt und Pein bezeichnen dich,\\
  du wirſt den Menſchen kennbar ſeyn,\\
  daß jeder, der von fern dich ſiehet,\\
  den Weg des Brudermörders fliehet.\\
  So ſtieg die Wolke donnernd auf.\\
  Nun nimmt er ſeinen Lauf in öde Gegenden.\\
  Ich geh mit ihm, will mit ihm weinen,\\
  bis daß dem Büßenden wird Gnad und Troſt erſcheinen.\\
  Ach, Eva, Vater, ſegne mich!

  \voice[Eva, Adam]
  Gott ſegne dich, daß er, durch dich erweckt,\\
  vor dem Verbrechen bebe, und wein und fleh,\\
  biß Gott ihm ganz vergebe.
\end{movement}

\begin{movement}{acheva}
  \voice[Hamiel]
  Ach Eva! komm und hilf!\\
  Die Mutter ſchläft ſehr tief.\\
  Sie gab mit Aug und Hand\\
  mir gar kein einzges Zeichen;\\
  da ich doch weinend bath,\\
  und Mutter! Mutter rief.

  \voice[Eva]
  Gott, ſteh ihr bey!\\
  Vielleicht kann ich ſie noch erreichen,\\
  bevor ſie ſtirbt.\\
  Allein – ſie iſt – ja, ſie iſt todt! –\\
  das Auge ſtarrt, – die Lippen sind erblaſſet! –\\
  welch eine Quaal, da Thirza mich verlaſſet!

  \voice[Adam]
  Was hör ich? – Thirza lebt nicht mehr? –\\
  gerechter Gott! – mir ſchwillt vor Angſt die Bruſt! –\\
  gieb Kraft bey ſolchen Plagen,\\
  ſonſt kann mein Vaterherz\\
  den Schmerzen nicht ertragen;\\
  er wird zu ſchwer!

  \voice[Eva]
  Wie iſt mir? welche dunkle Nacht\\
  benahm mir all Gefühl,\\
  und ſchloß mir beyde Augen?\\
  Gott! – ach! was habe ich gethan!\\
  ich hab den Tod, – ich hab dieß Übel aufgebracht.\\
  O die Unſchuldigen! – Herr! laß mich ſterben!

  \voice[Adam]
  Wie? Eva! faſſe dich!\\
  und eile dem Verderben\\
  durch allzubangen Gram nicht ſelbſten zu!\\
  Wir haben dieß, und noch weit mehr verſchuldet.\\
  Nur Gnade iſts, wenn Gott uns hier noch länger duldet.

  \voice[Sunam]
  Ach Bruder! welcher Schmerzentag\\
  ſtört heut auf einmal alle Ruh!
\end{movement}

\begin{movement}{oschmerz}
  \voice[Sunam]
  O Schmerz! Mein Vater iſt erblichen!

  \voice[Hamiel]
  O Quaal! Die Mutter iſt entwichen!

  \voice[beyde]
  Nun ſind wir vatermutterlos,\\
  ach, der Verluſt ist gar zu groß.

  \voice[Sunam]
  Nur du, o Adam!, du allein,

  \voice[Hamiel]
  Nur du, o Eva!, du allein,

  \voice[beyde]
  ihr könnet uns Verlaſſnen

  \voice[Sunam]
  ſtatt Abel Vater seyn.

  \voice[Hamiel]
  ſtatt Thirza Mutter ſeyn.
\end{movement}

\begin{movement}{derenkel}
  \voice[Adam]
  Der Enkel Klaggeſang\\
  durchdringt, erſchüttert mich,\\
  gräbt alle Wunden neu.\\
  Gott! unterſtütze mich!\\
  gieb meiner Gattinn Gnade!
\end{movement}

\begin{movement}{meinabel}
  \voice[Eva]
  Mein Abel, meine Thirza todt!\\
  ſie ſind dahin! –\\
  O Schmerz! Ach welcher Schade!

  \voice[Sunam]
  Mein Vater todt!

  \voice[Hamiel]
  Die Mutter todt!

  \voice[beyde]
  ſie ſind dahin! –\\
  O Schmerz! Ach welcher Schade!
\end{movement}

\begin{movement}{derherr}
  \voice[Adam]
  Der Herr allein, der Herr iſt Gott!\\
  er iſt allein Herr über Tod und Leben.\\
  Deßhalben klaget nicht!\\
  ſein Urtheil iſt gerecht;\\
  er kann dieß, was er nimmt, uns hundertfältig geben.\\
  Auch in der Strafe bleibt er Vater: ich der Knecht.\\
  Des Abels und der Thirza Grab\\
  ihr zarten Enkel! helfet mir\\
  mit Roſen und Kypreſſen zieren.\\
  Wo Abel liegt, bey dieſer Laube hier,\\
  will ich den Hauch, den ich vom Höchſten hab,\\
  ihr Enkel! unter euch zum Schöpfer wieder führen.
\end{movement}

% \begin{movement}{}
%   \voice[]
% \end{movement}
}

\eesScore

\end{document}
