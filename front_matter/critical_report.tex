\documentclass[parskip=full]{scrreprt}

\RedeclareSectionCommand[pagestyle=plain,afterskip=10pt plus 1pt]{chapter}
\setkomafont{chapter}{\mdseries\headingfont\fontsize{40}{40}\selectfont\color{black!80}}
\setkomafont{pageheadfoot}{\normalsize}

\def\pnumbox#1{#1\hspace*{8cm}}
\def\gobble#1{}
\DeclareTOCStyleEntry[
  indent=0pt,
  beforeskip=0pt,
  entryformat=\itshape,
  entrynumberformat=\textcolor{oldred},
  numwidth=2em,
  linefill=\hfill,
  pagenumberbox=\pnumbox,
  pagenumberformat=\itshape
]{tocline}{part}

\DeclareTOCStyleEntry[
  indent=2em,
  beforeskip=-\baselineskip,
  entrynumberformat=\textcolor{oldred},
  numwidth=2em,
  linefill=\hfill,
  pagenumberbox=\pnumbox
]{tocline}{section}

\DeclareTOCStyleEntry[
  indent=0pt,
  beforeskip=-\parskip,
  entrynumberformat=\gobble,
  entryformat=\ltseries,
  numwidth=2em,
  linefill=\hfill,
  pagenumberbox=\pnumbox,
  pagenumberformat=\ltseries
]{tocline}{subsection}


\usepackage{polyglossia}
\setdefaultlanguage{english}


\usepackage{fontspec}
\setmainfont{Source Sans Pro}[
  ItalicFont = Source Sans Pro Italic,
  BoldFont = Source Sans Pro Bold,
  BoldItalicFont = Source Sans Pro Bold Italic,
  FontFace = {lt}{n}{Source Sans Pro Light},
  FontFace = {lt}{it}{Source Sans Pro Light Italic},
  FontFace = {sb}{n}{Source Sans Pro Semibold},
  FontFace = {sb}{it}{Source Sans Pro Semibold Italic},
  Numbers = Proportional,
  Ligatures = Common
]
\DeclareRobustCommand{\ltseries}{\fontseries{lt}\selectfont}
\DeclareRobustCommand{\sbseries}{\fontseries{sb}\selectfont}
\DeclareTextFontCommand{\textlt}{\ltseries}
\DeclareTextFontCommand{\textsb}{\sbseries}
\newfontfamily\headingfont{Fredericka the Great}


\usepackage[pass]{geometry}
\newgeometry{twoside,inner=20mm,outer=40mm,top=20mm,bottom=40mm}


\usepackage{url}
\urlstyle{same}


\usepackage{microtype}
\microtypesetup{verbose=silent}


\usepackage{booktabs,array,longtable}
\newcolumntype{L}[1]{%
  >{\raggedright\let\newline\\\arraybackslash\hspace{0pt}}p{#1}%
}


\usepackage{graphicx}


\usepackage{xcolor}
\definecolor{oldred}{rgb}{.8313,0,0}


\usepackage{pdfpages}


\usepackage{scrlayer-scrpage}
\pagestyle{scrheadings}
\clearscrheadfoot
\cfoot[\thepage]{\thepage}
\pagenumbering{roman}

\usepackage{enumitem}
\setlist[description]{%
 labelindent=2em,%
 labelwidth=6.5em,%
 leftmargin=8.5em,%
 labelsep=0pt,
 first=\ltseries,%
 font=\normalfont\itshape\ltseries%
}
\def\lyrefitem#1{\item[\lyref{#1}]}


\makeatletter

\def\@firstofthree#1#2#3{#1}
\def\@secondofthree#1#2#3{#2}
\def\@thirdofthree#1#2#3{#3}
\def\@firstoffour#1#2#3#4{#1}
\def\@secondoffour#1#2#3#4{#2}
\def\@thirdoffour#1#2#3#4{#3}
\def\@fourthoffour#1#2#3#4{#4}
\def\Dotfill{\leavevmode\cleaders\hb@xt@ .75em{\hss .\hss }\hfill \kern \z@}

\def\lyrefnumber#1{\expandafter\@setref\csname r@#1\endcsname\@firstofthree{#1}}
\def\lyreftitle#1{\expandafter\@setref\csname r@#1\endcsname\@secondofthree{#1}}
\def\lyrefpage#1{\expandafter\@setref\csname r@#1\endcsname\@thirdofthree{#1}}

\def\lyrefgenrenumber#1{\expandafter\@setref\csname r@#1\endcsname\@firstoffour{#1}}
\def\lyrefgenregenre#1{\expandafter\@setref\csname r@#1\endcsname\@secondoffour{#1}}
\def\lyrefgenretitle#1{\expandafter\@setref\csname r@#1\endcsname\@thirdoffour{#1}}
\def\lyrefgenrepage#1{\expandafter\@setref\csname r@#1\endcsname\@fourthoffour{#1}}

\def\lyref#1{%
  \begingroup%
  \makebox[0pt][l]{\color{oldred}\lyrefnumber{#1}}\hspace*{2em}%
  \lyreftitle{#1}\Dotfill\lyrefpage{#1}%
  \endgroup%
}
\def\lyrefpart#1{%
  \begingroup%
  \makebox[0pt][l]{\sbseries\color{oldred}\lyrefnumber{#1}}\hspace*{2em}%
  \makebox[0pt][l]{\sbseries\lyreftitle{#1}}\hspace*{6.5em}%
  \hfill\sbseries\lyrefpage{#1}%
  \endgroup%
}
\def\lyrefsection#1{%
  \begingroup%
  \makebox[0pt][l]{\color{oldred}\lyrefgenrenumber{#1}}\hspace*{2em}%
  \makebox[0pt][l]{\ltseries\lyrefgenregenre{#1}}\hspace*{6.5em}%
  \lyrefgenretitle{#1}\Dotfill\lyrefgenrepage{#1}%
  \endgroup%
}
\InputIfFileExists{../tmp/lilypond.ref}{}{\InputIfFileExists{../lilypond.ref}{}{}}


\newcommand\fancytitlehead{
  \headingfont%
  \fontsize{70}{70}\selectfont%
  \textcolor{black!80}{%
    \makebox[0pt][l]{\@ifundefined{@shortname}{\@lastname}{\@shortname}.}%
  }\\[15pt]%
  \fontsize{55}{55}\selectfont%
  \makebox[0pt][l]{\@ifundefined{@shorttitle}{\@title}{\@shorttitle}.}%
}


\def\firstname#1{\def\@firstname{#1}}
\def\lastname#1{\def\@lastname{#1}}
\def\shortname#1{\def\@shortname{#1}}
\def\shorttitle#1{\def\@shorttitle{#1}}
\def\namesuffix#1{\def\@namesuffix{#1}}
\def\scoring#1{\def\@scoring{#1}}
\def\parts#1{\def\@parts{#1}}

\firstname{\relax}
\lastname{\relax}
\namesuffix{\relax}
\scoring{\relax}
\parts{\relax}


\def\maketitle{%
\begin{titlepage}%
  \Large%
  {\@titlehead}%
  \vfill%
  {\strut\@firstname}\\%
  {\sbseries\color{oldred}\strut\@lastname}\\%
  {\strut\@namesuffix}%
  \vfill%
  {\sbseries\@title}\\%
  {\@subtitle}\\[\baselineskip]%
  {\itshape\@scoring}%
  \vfill%
  {\itshape\@parts}\hspace*{\fill}\raisebox{0pt}[0pt][0pt]{\includegraphics{ees_logo}}%
\end{titlepage}%
}


\newif\iftemplate\templatetrue
\newif\ifprintreport\printreportfalse
\directlua{
scores = {
  ob1 = "Oboe I",
  ob2 = "Oboe II",
  ottoni = "Clarino I, II in C\string\\newline Timpani in C–G",
  vl1 = "Violino I",
  vl2 = "Violino II",
  vla = "Viola",
  coro = "Coro",
  org = "Organo",
  b = "Bassi",
  full_score = "Full Score"
}

last_arg = arg[\string#arg]
texio.write("Last argument: " .. last_arg)
if not (scores[last_arg] == nil) then
  tex.print("\string\\def\string\\lypdfname{" .. last_arg .. "}")
  tex.print("\string\\parts{" .. scores[last_arg] .. "}")
  if (last_arg == "full_score") then
    tex.print("\string\\printreporttrue")
  end
end
}

\@ifundefined{lypdfname}{%
  \templatefalse
  \printreporttrue
  \parts{Draft}
}{\templatetrue}

\makeatother



\begin{document}
\frenchspacing

\titlehead{\fancytitlehead}
\firstname{\makebox[9cm][l]{Johann Heinrich\hfill Johann Michael}}
\lastname{\makebox[9cm][l]{Rolle\hfill \&\hfill Haydn}}
\shortname{Rolle \& Haydn}
\title{Der Tod Abels}
\subtitle{WacR I:7, MH 271}
\scoring{S, A, T, B (solo), S, A, T, B (coro), 2 ob, 2 clno, timp, 2 vl, vla, b, org}
\maketitle


\thispagestyle{empty}

\vspace*{\fill}

\raisebox{-4mm}{\includegraphics{byncsaeu}}\hspace*{1em}Wolfgang Esser-Skala, 2021

© 2021 by Wolfgang Esser-Skala. This edition is licensed under the Creative Commons Attribution-NonCommercial-ShareAlike 4.0 International License. To view a copy of this license, visit \url{http://creativecommons.org/licenses/by-nc-sa/4.0/}.

Music engraving by LilyPond 2.22.0 (\url{http://www.lilypond.org}).\\
Front matter typeset with Source Sans Pro and Fredericka the Great.

\textit{First version, November 2021}

\vspace*{2cm}

\ifprintreport
\chapter*{Critical Report}

This edition bases upon
\begin{enumerate}\raggedright
  \item a copy in the Austrian National Library (siglum Mus.Hs.15529; no RISM entry); digital version available at \url{http://data.onb.ac.at/rec/AC14290384};
  \item a copy in the Bonn University and State Library (siglum Ec 149.12; see also RISM ID 450064107); digital version available at \url{https://digitale-sammlungen.ulb.uni-bonn.de/urn/urn:nbn:de:hbz:5:1-298443};
  \item a contemporary print (Breitkopf, Leipzig, 1771) in the Bayerische Staatsbibliothek München (siglum 4 Mus.pr. 1078 ; see also RISM ID 990055403); digital version available at \url{https://mdz-nbn-resolving.de/urn:nbn:de:bvb:12-bsb11140540-5}; and
  \item Haydn’s autograph manuscript in the Bayerische Staatsbibliothek München (siglum Mus.ms. 3736; see also RISM ID 455016820); digital version available at \url{https://mdz-nbn-resolving.de/urn:nbn:de:bvb:12-bsb00084995-7}.
\end{enumerate}

In general, this edition closely follows the manuscripts. Any changes that were introduced by the editor are indicated by italic type (lyrics, dynamics and directions), parentheses (expressive marks and bass figures) or dashes (slurs and ties). Accidentals are used according to modern conventions. Asterisks denote changes that are clarified in the detailed remarks below.\footnote{Abbreviations: A, alto; B, bass; cor, horn; fag, bassoon; fl, flute; fond, fondamento; Ms, manuscript; ob, oboe; r, rest; S, soprano; T, tenor; timp, timpani; vl, violin; vla, viola.}

\bigskip

\begin{longtable}{lll L{10cm}}
  \toprule
  \itshape Mov. & \itshape Bar & \itshape Staff & \itshape Note \\
  \midrule \endhead
  1.1 & 6  & ob, fag, vl 2 & grace note missing in (1) \\
      & 7  & ob, fag, vl & grace note missing in (1) \\
      & 10 & ob 1        & 1st half note in Ms as′2 \\
      & 16 & ob 2, fag 1, vl 2 & grace note missing in (1) \\
      & 17 & A           & grace note missing in (1) \\
      & 33 & ob 2, vl 2  & grace note missing in (1) \\
      & 34 & A           & grace note missing in (1) \\
      & 50 & fag         & grace note missing in (1) \\
      & 51 & A           & grace note missing in (1) \\
      & 52 & A           & 1st half note in (1): g′2 \\
  \bottomrule
\end{longtable}


This edition has been compiled and checked with utmost diligence. Nevertheless, errors and mistakes cannot be totally excluded. Please report any errors and mistakes to \url{wolfgang@esser-skala.at} or create an issue or pull request on the edition’s GitHub page \url{https://github.com/skafdasschaf/rolle-der-tod-abels-WacR-I-7}. Your help will be greatly appreciated.

\bigskip
\textit{Koppl, November 2021\\
Wolfgang Esser-Skala}

\cleardoublepage
\chapter*{Contents}

\lyrefpart{erstertheil}

\lyrefsection{lobtden}

\begin{description}
  \item[Coro]
  Lobt den Herrn! Lobt den Herrn!\\
  Die Morgenſonne weckt die Flur aus ihrer Ruh;\\
  und der ganzen Schöpfung Wonne\\
  ſtrömt verjüngt uns wieder zu.\\
  ~\\
  Lobt den Herrn! Lobt den Herrn!\\
  In frühen Düften lobet ihn der Blumen Flor;\\
  auf den Wipfeln, in den Lüften\\
  ſinget ihn der Vögel Chor.\\
  ~\\
  Lobt den Herrn! Lobt den Herrn!\\
  Aus ſeiner Höhle brüllt das Wild ihm ſeinen Dank.\\
  O! vor allen, meine Seele,\\
  tön ihm früh dein Lobgeſang!
\end{description}

\lyrefsection{siesingen}

\begin{description}
  \item[Basso]
  Sie ſingen!\\
  Ha, gewiß ein neues Lied des Lieblings,\\
  dem allein die heitre Freude blüht,\\
  der müßig bey der Heerde ſitzet,\\
  wenn dieſe Stirn von ſchwerer Arbeit ſchwitzet.\\
  Seh ich nicht in der Laub auch Adam?\\
  Wie entzückt umarmt er ihn!\\
  Mit welcher Liebe blickt ihn Eva an!\\
  Wie ihre Herzen ſich ergießen!\\
  Wie ihre Freudenthränen fließen!\\
  Ach, immer heller wird mein dunkles Traumgeſicht.\\
  Ich ſeh, ich ſeh, es täuſcht mich nicht:\\
  Mein iſt der Fluch und ſein der Seegen.\\
  Sah ich im Traum auf Blumenwegen\\
  nicht alle Kinder Abels gehn?\\
  In ihrem Thal war die Natur nur ſchön.\\
  In dunklen Schatten grüner Lauben,\\
  in Feigenhaynen, unter Trauben,\\
  umkränzt mit Roſen ſaßen ſie,\\
  und goldne Frucht fiel ohne Müh in ihren Schoos,\\
  nur ſanfte Lieder ſangen ſie.\\
  Ein dornicht Feld, ein Thal, wo Armuth wohnet,\\
  und Seegen nie den Fleiß belohnet,\\
  war meiner Kinder Theil.\\
  Und ach! mit welcher Quaal,\\
  erinnert ſichs mein Geiſt!\\
  in dieſes Dornenthal fiel in der Nacht\\
  die Schaar von Abels Söhnen.\\
  Noch ſeh ich meine Kinder höhnen,\\
  noch ſchallet ihr Geſchrey\\
  mir ſchrecklich in mein Ohr,\\
  noch ſteigt vor mir die Flamme hoch empor\\
  von Hütten, die der Feind entzündet,\\
  noch ſeh ich, wie man ſie als Sclaven bindet,\\
  hört nicht ihr Flehen, nicht ihr Schreyn,\\
  und führt ſie weg, dienſtbar zu ſeyn.
\end{description}

\cleardoublepage
\fi

\iftemplate
\includepdf[pages=-]{../tmp/\lypdfname.pdf}
\fi

\end{document}
