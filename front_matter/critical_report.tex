\documentclass[tocstyle=ref-genre]{ees}

\firstname{\makebox[9cm][l]{Johann Heinrich\hfill Johann Michael}}
\lastname{\makebox[9cm][l]{Rolle\hfill \&\hfill Haydn}}
\shortname{Rolle \& Haydn}

\begin{document}

\eesTitlePage

\eesCriticalReport{
  –    & –   & org         & Bass figures appear in the following movements (bars in parentheses): 1.2, 1.4, 1.6 (20), 1.8, 1.10, 1.12, The remaining bass figures were added by the editor. \\
  1.1  & 6   & ob, fag, vl 2 & grace note missing in (1) \\
       & 7   & ob, fag, vl & grace note missing in (1) \\
       & 10  & ob 1        & 1st half note in Ms as′2 \\
       & 16  & ob 2, fag 1, vl 2 & grace note missing in (1) \\
       & 17  & A           & grace note missing in (1) \\
       & 33  & ob 2, vl 2  & grace note missing in (1) \\
       & 34  & A           & grace note missing in (1) \\
       & 50  & fag         & grace note missing in (1) \\
       & 51  & A           & grace note missing in (1) \\
       & 52  & A           & 1st half note in (1): g′2 \\
  1.4  & 51  & vla         & 1st half note in (1): es′2 \\
  1.7  & 22  & fl, fag     & rhythm of 5th eighth in (1): 16–16 \\
       & 29  & fag 2, vl, Kain & grace note missing in (1) \\
       & 31  & fl 2        & grace note missing in (1) \\
       & 34  & fl 1        & 1st quarter in Ms: g″16–es′′′16 \\
       & 41  & fl 2        & grace note missing in (1) \\
  1.9  & 16  & cor 2       & grace note missing in (1) \\
       & 20  & cor         & grace note missing in (1) \\
       & 64  & vl 2        & grace note missing in (1) \\
       & 116 & vla         & bar in (1): g′4–g′8 \\
  1.10 & –   & org         & upper voice after (1), lower voice after (3) \\
  1.11 & 8   & vl 2        & bar in (1): e″4.–d″8 \\
       & 17  & vl 2        & bar in (1): e′4.–d′8 \\
  1.13 & 82  & vla         & grace note missing in (1) \\
       & 86  & vla         & grace note missing in (1) \\
       & 163 & vl 2        & grace note missing in (1) \\
       & 165 & vl 2        & grace note missing in (1) \\
       & 215 & vl 2        & 1st eighth in (1): fis″8 \\
       & 229 & vl 1        & grace note missing in (1) \\
       & 233 & vl          & grace note missing in (1) \\
}

\eesToc{
% Thirza (S), Eva (A), Mehala (A), Abel (T), Adam (B), Kain (B)

\lyrefpart{erstertheil}

\lyrefsection{lobtden}

\begin{description}
  \item[Coro]
  Lobt den Herrn! Lobt den Herrn!\\
  Die Morgenſonne weckt die Flur aus ihrer Ruh;\\
  und der ganzen Schöpfung Wonne\\
  ſtrömt verjüngt uns wieder zu.\\
  ~\\
  Lobt den Herrn! Lobt den Herrn!\\
  In frühen Düften lobet ihn der Blumen Flor;\\
  auf den Wipfeln, in den Lüften\\
  ſinget ihn der Vögel Chor.\\
  ~\\
  Lobt den Herrn! Lobt den Herrn!\\
  Aus ſeiner Höhle brüllt das Wild ihm ſeinen Dank.\\
  O! vor allen, meine Seele,\\
  tön ihm früh dein Lobgeſang!
\end{description}

\lyrefsection{siesingen}

\begin{description}
  \item[Kain]
  Sie ſingen!\\
  Ha, gewiß ein neues Lied des Lieblings,\\
  dem allein die heitre Freude blüht,\\
  der müßig bey der Heerde ſitzet,\\
  wenn dieſe Stirn von ſchwerer Arbeit ſchwitzet.\\
  Seh ich nicht in der Laub auch Adam?\\
  Wie entzückt umarmt er ihn!\\
  Mit welcher Liebe blickt ihn Eva an!\\
  Wie ihre Herzen ſich ergießen!\\
  Wie ihre Freudenthränen fließen!\\
  Ach, immer heller wird mein dunkles Traumgeſicht.\\
  Ich ſeh, ich ſeh, es täuſcht mich nicht:\\
  Mein iſt der Fluch und ſein der Seegen.\\
  Sah ich im Traum auf Blumenwegen\\
  nicht alle Kinder Abels gehn?\\
  In ihrem Thal war die Natur nur ſchön.\\
  In dunklen Schatten grüner Lauben,\\
  in Feigenhaynen, unter Trauben,\\
  umkränzt mit Roſen ſaßen ſie,\\
  und goldne Frucht fiel ohne Müh in ihren Schoos,\\
  nur ſanfte Lieder ſangen ſie.\\
  Ein dornicht Feld, ein Thal, wo Armuth wohnet,\\
  und Seegen nie den Fleiß belohnet,\\
  war meiner Kinder Theil.\\
  Und ach! mit welcher Quaal,\\
  erinnert ſichs mein Geiſt!\\
  in dieſes Dornenthal fiel in der Nacht\\
  die Schaar von Abels Söhnen.\\
  Noch ſeh ich meine Kinder höhnen,\\
  noch ſchallet ihr Geſchrey\\
  mir ſchrecklich in mein Ohr,\\
  noch ſteigt vor mir die Flamme hoch empor\\
  von Hütten, die der Feind entzündet,\\
  noch ſeh ich, wie man ſie als Sclaven bindet,\\
  hört nicht ihr Flehen, nicht ihr Schreyn,\\
  und führt ſie weg, dienſtbar zu ſeyn.
\end{description}

\lyrefsection{ichelend}

\begin{description}
  \item[Kain]
  Ich elend! elend meine Kinder!\\
  Gewitter Gottes, trefft mich Sünder!\\
  Abgrund, eröffne dich! verſchlinge mich!\\
  Umſonſt iſt dieſes Flehen,\\
  der Richter läßt es nicht geſchehen.\\
  Elender, du mußt elend ſeyn!\\
  ſo waffne dich, und ſey es nicht allein!
\end{description}

\lyrefsection{seymir}

\begin{description}
  \item[Adam]
  Sey mir gegrüßt, mein erſtgebohrner Sohn!\\
  ach, daß ich dieſen Trauerton\\
  von deinen Lippen nicht gehöret,\\
  der Pein in meinem Buſen nähret!\\
  Die führt mich zu dir, dieſe Pein.

  \item[Kain]
  Nicht Liebe, die gehört dem Abel nur allein.

  \item[Adam]
  Ja Kain, zärtliche beſorgte Liebe.\\
  Was nähreſt du für finſtre Triebe,\\
  für ſchwarzen Groll in deiner Bruſt?\\
  Groll gegen den, der unſre Luſt\\
  durch ſeinen heitern Reiz der Jugend,\\
  durch ſeine Andacht, ſeine Tugend\\
  und ſeine ſanften Lieder iſt!\\
  O du! der du mein Erſtgebohrner biſt!\\
  Mein Sohn, mein Kain!\\
  quäle mit dieſem Ungeſtüm nicht deine Seele.\\
  Lieb ihn, wie er dich liebt,\\
  erheitre dein Geſicht.

  \item[Kain]
  Wie Abel lächeln kann ich nicht,\\
  gebieten kann ich nicht dem Ernſt,\\
  der euch verdrießet,\\
  daß er in Thränen ſanft zerfließet.

  \item[Adam]
  Das iſt kein männlicher, kein ernſter Sinn,\\
  nein, Unzufriedenheit und Gram reißt dich dahin,\\
  dein freudenloſes finſteres Betragen zeigt es,\\
  und die itzt ausgeſtoßnen Klagen.

  \item[Kain]
  Vielleicht ſind auch mehr Freuden noch für dich.\\
  Die größte Laſt des Fluches fiel auf mich,\\
  den Erſtgebohrnen;\\
  nur ich bin verlohren,\\
  nur ich bin vom Weibe zum Elend gebohren!

  \item[Adam]
  Ach Sohn! was hört mein Ohr!\\
  Du läſterſt! Kein Geſchöpf ruft Gott zum Elend hervor.\\
  Laß dich Vernunft und Tugend rühren,\\
  ſie werden Freuden dir in deine Seele führen.\\
  Mein Sohn! ach höre mich!\\
  bey dieſen Thränen bitt ich dich,\\
  lieb Abeln! du wirſt uns entzücken,\\
  und wie wird er an ſeine Bruſt dich drücken!

  \item[Kain]
  Ich haß ihn nicht. Doch, was euch alle rührt,\\
  der Weichlichkeit, die mir dein Herz entführt,\\
  die Thränen euch entlockt,\\
  der hab ich Haß geſchworen,\\
  dadurch haſt du das Paradies verlohren.
\end{description}

\lyrefsection{owort}

\begin{description}
  \item[Adam]
  O Wort, dafür mein Geiſt erzittert,\\
  das wie ein Donner mich erſchüttert.\\
  O Vorwurf voller Todespein,\\
  mit tauſend Stacheln gräbt er\\
  in mein Herz ſich ein!\\
  Welche Ahndung faſſet mich!\\
  Also werden Enkel dich\\
  in der Erde fernſten Tagen,\\
  Adam, Adam, laut verklagen,\\
  Rettung aus dem Elend ſuchen,\\
  aber keine Rettung ſehn!\\
  über deinem Haupte ſtehn\\
  und dir erſten Sünder fluchen!
\end{description}

\lyrefsection{wieseufzet}

\begin{description}
  \item[Kain]
  Wie ſeufzet er!\\
  wie kläglich ringt er\\
  über ſeinem Haupt die Hände!\\
  Mein nagender qualvoller Vorwurf\\
  dringt ihm tief ins Herz.\\
  Ach, Kain! ach, Kain!\\
  wende von diesem Ungeſtüm\\
  dich zur Vernunft zurück.\\
  Kannſt du des frommen Vaters Blick\\
  voll ſolcher Seelenangſt ertragen,\\
  ſein Seufzen hören, und ſein Klagen!\\
  Noch biſt du nicht ganz deines Elends Raub,\\
  eil, und wirf dich vor ihm hin in den Staub!
\end{description}

\lyrefsection{meinvater}

\begin{description}
  \item[Kain]
  Mein Vater, ach! verzeihe!\\
  Sieh dieſe Thränen meiner Reue,\\
  ich ſehe das, was ich gethan,\\
  mit Schauer und Entſetzen an.\\
  Ach Vater, fluche nicht dem Sohne,\\
  von dir zu des Allmächtgen Throne,\\
  zu meinem Bruder will ich gehn,\\
  Vergebung zu erflehn.
\end{description}

\lyrefsection{ogott}

\begin{description}
  \item[Adam]
  O Gott! mein Blick ſchaut dankbar zu dir auf!\\
  Heut erhöreſt du mein Sehnen.\\
  Mein Kain! o, mein Sohn, ſteh auf!\\
  Mit Wohlgefallen ſieht Gott dieſe Thränen.\\
  Seyd mir geſegnet, frohe Stunden!\\
  Ich habe meinen Erſtgebohrnen wieder funden.\\
  Er bringt mit dieſem thränenvollen Blick\\
  uns Freuden, Fried und Ruh zurück.\\
  O Sohn! wie wird Gott deiner ſich erbarmen,\\
  dich zögre nicht, komm, laß den Bruder dich umarmen!

  \item[Abel]
  Du liebeſt mich, mein Bruder!\\
  du liebſt mich! ach! ſag es mir,\\
  daß ich von deinen Lippen es vernehme!

  \item[Kain]
  Ich liebe dich!\\
  ja, ich Elender ſchäme des Unrechts mich,\\
  daß ich dir dieſes Herz verſagt,\\
  die Ruh ſo lang von euch verjagt,\\
  mit Unmuth eure Tag erfüllet,\\
  und mich in Trübſinn eingehüllet.\\
  Schnell hebt ſich meine Seel empor,\\
  und geht aus ihrer Nacht hervor.\\
  Mein Bruder! du kannſt mir vergeben,\\
  und ſieheſt nicht zurück in das vergangne Leben.
\end{description}

\lyrefsection{wennderjunge}

\begin{description}
  \item[Abel]
  Wenn der junge Tag erwacht,\\
  ſo verſchwinden alle Sorgen,\\
  die ein leichter Traum am Morgen\\
  auf dem Lager uns gemacht.\\
  O Kain, Kain! mein Entzücken\\
  vermag ich dir nicht auszudrücken,\\
  der Ton erſtirbt für ſüße Luſt;\\
  nur drücken kann ich dich an dieſe Bruſt.
\end{description}

\lyrefsection{okinder}

\begin{description}
  \item[Eva]
  O Kinder! ſeit das Paradies verſchwunden,\\
  hab ich nicht ſolche Freud empfunden,\\
  als dieſer Anblick mir gewährt,\\
  da Fried und Eintracht wiederkehrt.\\
  Ach, Adam! die, die wir erzeugten,\\
  die lieben ſich!\\
  Nichts iſt meiner Wonne gleich.\\
  Umarmt, geliebten Kinder, euch.\\
  Die Thränen, die aus euren Augen fließen,\\
  will ich von euren Wangen küßen.
\end{description}

\lyrefsection{achschwester}

\begin{description}
  \item[Mehala]
  Ach Schweſter! ſing in meine Lieder!\\
  heut kommt die ſanfte Ruh mir wieder.

  \item[Thirza]
  Mehala! wie der Lenz erquicket,\\
  hat dieſer Anblick mich entzücket.

  \item[Mehala]
  Du biſt mir ſchöner nun, Natur!

  \item[Thirza]
  Du blühſt mir lieblicher, o Flur!

  \item[Mehala]
  Dein Licht iſt heller, ſtiller Mond!

  \item[both]
  Nun Ruh in unſern Hütten wohnt.

  \item[Thirza]
  O paradieſisch große Freude,\\
  die beſten Blumen laßt uns beyde\\
  in unſre Laube ſtreun.

  \item[Mehala]
  O paradieſisch ſchönes Leben!\\
  die beſte Frucht vom Baum und Reben\\
  ſoll unſer Herz erfreun.

  \item[both]
  Und dieſer Tag ein Feſt uns ſeyn.
\end{description}

\lyrefsection{meinbruder}

\begin{description}
  \item[Abel]
  Mein Bruder! unſerm Gotte, der uns liebt,\\
  der meinen Bruder heut mir wiedergiebt,\\
  will ich an meinem Altar danken.\\
  Haſt du nicht auch, Geliebter, den Gedanken?\\
  Willſt du nicht auch zu deinem Altar gehn?\\
  Das ſchönſte Lamm hab ich erſehn,\\
  dem Herrn zum Opfer anzuzünden.\\
  Auch du, mein Bruder, wirſt ein Opfer finden,\\
  dem Herren angenehm.\\
  Verſiegle beym Altar den Bund,\\
  der ſtets der Wunſch von meinem Herzen war.

  \item[Kain]
  Ich will es thun,\\
  und auch dem Herrn ein Opfer bringen.\\
  Zwar wenig kann ich nur erzwingen,\\
  das, was des Feldes Armuth giebt.

  \item[Abel]
  Du weißt, Geliebter, daß Gott nicht das Opfer liebt,\\
  den Opfernden liebt er;\\
  er achtet nicht auf das Lamm, das man ihm ſchlachtet,\\
  nicht auf die Frucht, die eine Flamm verzehrt,\\
  wenn Lieb und Andacht nur im Herzen ihn verehrt.

  \item[Mehala]
  Geliebter, zürne nicht! Ich ſeh die Zeichen\\
  des finſtern Grams zurück auf deine Stirne ſchleichen,\\
  ach, kämpfe, daß er nicht dein Herz\\
  erfülle, uns aufs neu zum Schmerz!

  \item[Kain]
  Sey ruhig; er ſoll nicht mein Herz erfüllen,\\
  mit Dunkel eure Tage nicht umhüllen.\\
  Zwar ihr beleidigt mich ins Angeſicht\\
  mit eurem Uebermas von Freuden:\\
  So laſterhaft war Kain nicht,\\
  als ihr dadurch mich macht.\\
  Doch, ſchon gewohnt zu leiden,\\
  erduld ich es.\\
  Wer mit des Feldes Arbeit ringt,\\
  bezwingt auch leichten Gram,\\
  der zu dem Herzen dringt.\\
  Mein Opfer ſoll auf zu dem Herren wallen,\\
  er thu mit mir nach ſeinem Wohlgefallen.

  \item[Adam]
  Mein Sohn! laß es ein frohes Opfer ſeyn.\\
  Schau, alles ruft dich auf, dich mit uns zu erfreun.\\
  Mit Unmuth iſt vor Gott von uns niemand erſchienen;\\
  wir ſollen ihm mit Freuden dienen.
\end{description}

\lyrefsection{frohgeht}

\begin{description}
  \item[Mehala]
  Froh geht dir die Sonne auf,\\
  froh vollendet ſie den Lauf!\\
  Feld und Flur im Blumenkleide\\
  und ſelbſt Arbeit giebt uns Freude.\\
  Opfert mit dem frohſten Triebe,\\
  und verſiegelt vor dem Herrn\\
  euren neuen Bund der Liebe,\\
  frohe Herzen hört er gern.
\end{description}
}

\eesScore

\end{document}
